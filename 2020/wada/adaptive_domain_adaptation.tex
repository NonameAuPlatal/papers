\documentclass[12pt,a4paper,twoside]{report}
\usepackage[left=3cm,right=3cm,top=2cm,bottom=3cm]{geometry}
\usepackage{pagenote}
\usepackage{algorithm}
\usepackage{algorithmic}
\usepackage{graphicx}
\usepackage[utf8]{inputenc}	% Para caracteres en español
\usepackage{amsmath,amsthm,amsfonts,amssymb,amscd}
\usepackage{multirow,booktabs}
\usepackage[table]{xcolor}
\usepackage{fullpage}
\usepackage{lastpage}
\usepackage{enumitem}
\usepackage{fancyhdr}
\usepackage{mathrsfs}
\usepackage{wrapfig}
\usepackage{setspace}
\usepackage{calc}
\usepackage{color,soul}
\usepackage{multicol}
\usepackage{cancel}
\usepackage[retainorgcmds]{IEEEtrantools}
\usepackage{xcolor}
\colorlet{shadecolor}{orange!15}
\parindent 0in
\parskip 12pt
\geometry{margin=1in, headsep=0.25in}
\theoremstyle{definition}
\newtheorem{defn}{Definition}
\newtheorem{reg}{Rule}
\newtheorem{exer}{Exercise}
\newtheorem{note}{Note}
\usepackage{listings}
\usepackage{spverbatim}
\usepackage{fancyvrb}
\usepackage{hyperref}
\usepackage{float} 
\usepackage{natbib}
\usepackage{tikz}
\usepackage{pgfplots}
\usepackage{pgfplotstable}
\usepackage{soulutf8}
\usepackage{color,soul}
\usepackage{xcolor}
\usepackage{tabularx}
\newtheorem{lemma}{Lemma}[section]
\newtheorem{theorem}{Theorem}[section]
\newtheorem{proposition}{Proposition}[section]
\newtheorem{definition}{Definition}[section]
\DeclareMathOperator*{\argmax}{argmax}
\usepackage[draft]{todo}
\newcommand{\fyTodo}[1]{\Todo[FY:]{\textcolor{orange}{#1}}}
\newcommand{\fyTodostar}[1]{\Todo*[FY:]{\textcolor{orange}{#1}}}
\newcommand{\fyDone}[1]{\done[FY]\Todo[FY:]{\textcolor{orange}{#1}}}
\newcommand{\fyDonestar}[1]{\done[FY]\Todo[FY:]{\textcolor{orange}{#1}}}
\pgfkeys{
    /tr/rowfilter/.style 2 args={
        /pgfplots/x filter/.append code={
            \edef\arga{\thisrow{#1}}
            \edef\argb{#2}
            \ifx\arga\argb
            \else
                \def\pgfmathresult{}
            \fi
        }
    }
}
\usepackage{mathtools,xparse}
\DeclarePairedDelimiter{\abs}{\lvert}{\rvert}
\DeclarePairedDelimiter{\norm}{\lVert}{\rVert}
\NewDocumentCommand{\normL}{ s O{} m }{%
  \IfBooleanTF{#1}{\norm*{#3}}{\norm[#2]{#3}}_{L_1(\Omega)}%
}

\begin{document}
\begin{center}
{\LARGE \bf Combination of generic and specific}\\
\end{center}
\setlength{\belowdisplayskip}{8pt} \setlength{\belowdisplayshortskip}{8pt}
\setlength{\abovedisplayskip}{8pt} \setlength{\abovedisplayshortskip}{8pt}
\setlist{nosep}
\setlength{\parskip}{0.1cm}
\setlength{\parindent}{1em}
\section*{Word Level Adaptive Domain Adaptation}
Let $h$ the output of the pretrained encoder, the adapted representation for domain $k$ will be $h' = h + a_k(h)$, meaning that all words in all sentences for domain $k$ will use the adapter module $a_k$.  A ``gated'' version uses $h' = h * (1-z_k) + a_k(h) * z_k$\footnote{More precisely, $a_k(h) = \sum_{l} a_{kl}(h)$, where the summation runs over layers, see Figure XXX...}\fyTodo{$a_k(h)=$ somme tous les composants résiduels} where $z_k$ is also a function of $h$ taking values in $[0,1]$. More precisely, $h$ and $h'$ are sequences of context vectors and the combination is performed element-wise, yielding:
\begin{equation}
  h'(w) = h(w) * (1 - z_k(h(w))) + a_k(h(w)) * z_k(h(w)). \label{eq:gated-residual}
\end{equation}

In Word Level Adaptive Domain Adaptation, $z_k(h(w))$ is designed to reflect the ``topicality'' of  word $w$ is in domain $k$: the more likely $w$ is in domain $k$, the larger $z_k(h(w))$ is. The word-level adaptive domain adaptation aims to reduce the variation caused by $a_k$ to the adapted context vector $h'$ (compared to $h$) for words that are not typical of domain $k$. For a domain, there are 2 type of non-topical words: out-of-domain words that rarely or never appear in the in-domain texts; or frequent words that appear equally frequently in many domains. \fyTodo{This is like a poor Tf-idf - can be because tf is small or because idf is small} By reusing a learned generic representation for the non-topical words, we can at least bound the risk of poor predictions in case of out-of-domain words by the risk of the learned generic model (see below).

\section*{Training process}
The training process comprises thre main steps:
\begin{itemize}
\item Pretraining a generic model with a mixed corpora.
\item Training a domain classifier on top of the encoder and decoder; during this step, the parameters of the generic model are frozen. This model computes the posterior domain probability $p(k|h(w))$ for each word $w$ based on the representation computed by the last layer. \fyTodo{Which layer, which classifier ?}
\item Training parameters of adapters with in-domain data separately while freezing the parameters of the generic model and of the domain classifiers.
\end{itemize}
During the inference, $z_k$ is used to regulate the strength of the adapter module as suggested in equation~\ref{eq:gated-residual}.

\section*{Weakness of the method}
<<<<<<< HEAD
The new method depends essentially to the quality of the classifier. Moreover, the classifier can fail to detect in-domain words in the case of polysemous words. For example, the word "joint" in joint stereo, joint sprain and joint project has 2 different translations as "joint" and "articulaire" in french. In the unbalanced dataset as the mixed corpora (6 in-domain corpora), the classifier somehow fails to predict IT as the most probable domain of "joint stereo".
=======
This method depends crucially to the quality of the classifier $p(k|h(w))$. Moreover, the classifier can fail to detect in-domain words in the case of polysemous words. For example, the word "joint" in ``joint stereo'', ``joint sprain'' and ``joint project'' correspond to multiple translations such as "joint" and "articulaire" in French. In the unbalanced dataset as the mixed corpora (6 in-domain corpora),  "joint stereo" \fyTodo{unfinished sentence}
>>>>>>> 5d2d2fe949df5f0f8a8e2b205347af1bc0ebb767

\section*{Analysis on the risk of new method}

The combination of a generic representation and a residual adapter is defined as follows:
\begin{equation}
  h' = h + a_k(h) \label{eq:residual-adapter}
\end{equation}

where $h$ is the output of shared ``generic'' layers, i.e $h=M(x,y)$, $M: \mathbb{X} \times \mathbb{Y} \rightarrow \mathbf{\Omega}\subset \mathbb{R}^{d_{model}}$, $a_k$ is the residual adapter defined as a function $a_k: \mathbb{R}^{d_{model}} \rightarrow \mathbb{R}^{d_{model}}$.

The poor performance of fine-tuned models with a residual layer happens as the value of $a_k(h)$ of an out-of-domain example $(x^{adv}, y^{adv})$ is adversarial to the model. Indeed, by fine-tuning the adapter with only in-domain data, $a_k$ never encounters adversarial examples, which makes it error-prone in the face of out-of-domain examples. The intuition of wada is that we control the weight of $a_k(h)$ in the combination, we can hope to reduce the wrong belief of $a_k$ on adversarial examples. To implement this idea, we  investigate a convex combination of the generic and the residual layers, and define the adapted representation to be:

\begin{equation}
h' = h * (1-f(z_k)) + a_k(h) * f(z_k), 
\label{eq:2}
\end{equation}

\noindent{}where
\begin{itemize}
\item $z_k = p(d=k | h=M(x,y)) = p_k(h)$ with $p: \mathbf{\Omega} \rightarrow \mathbf{\Delta}^{K}$\fyTodo{Attention à $x$ et $y$}
\item f is an activation function $f: [0,1] \rightarrow [0,1]$ 
\end{itemize}

We define some important constants:
\begin{itemize}
\item Domain inseparability:  \fyTodo{Esperance de classification error - is it }
  $$
  \mathbb{\varepsilon}_{insep}(\theta, p) = \sum_{k' \neq k} \mathbb{E}_{x,y \sim \mathcal{D}_{k'}}(p(k|h),
  $$
\item Risk of the generic model: $\mathbb{R} = \mathbb{E}_{(x,y) \sim \mathcal{D}}(l(M(x,y_{<t}), y_t))$ \fyDone{wrt to which distribution ?}
\item $\mathbf{\Omega_{k,p_{0}}} = \lbrace h | z_k = p_\theta(k|h) < p_0\rbrace$.
\item Risk of the adapted model for the in-domain test set:
  $$R_k = \mathbb{E}_{x,y \sim \mathcal{D}_k} [l(h * (1 - f(z_k)) + a_k(h) * f(z_k), y)]$$
\end{itemize}

We also have two assumptions on the regularity of the activation and the loss functions:
\begin{itemize}
\item $a_k$ is bounded in sense of $\normL{a_k(h)} < M$
\item the loss function is bounded in sense of $l(h * (1-f(z_k)) + a_k(h) * f(z_k), y) < A$ \fyTodo{why not $l(h,y) < A$ ?}
  $\forall x,y$
\end{itemize}
We then evaluate the risk of the model adapted to domain $k$ with residuals $a_k$ for arbitrary examples from $K$ domains:
\begin{equation}
\begin{split}
  R(M,a_k) &= \mathbb{E}_{x,y}[l(h * (1-f(z_k)) + a_k(h) * f(z_k),y)] \\
  &= \mathbb{E}_{x,y}[l(h * (1-f(z_k)) + a_k(h) * f(z_k),y)\mathbf{1}_{\mathbf{\Omega_{k,p_{0}}}}(h)] + \\
  &+ \mathbb{E}_{x,y}[l(h * (1-f(z_k)) + a_k(h) * f(z_k),y) \mathbf{1}_{\mathbf{\Omega_{k,p_{0}}^{c}}}(h)]
\end{split}
\end{equation}

By designing a suitable activation function, we can map $f: [0,p_{0}] \rightarrow [0,\delta]$. Then with $\delta$ is small enough we can have an approximation of the first term:
\begin{equation}
\mathbb{E}_{x,y}[l(h * (1-f(z_k)) + a_k(h) * f(z_k),y)\mathbf{1}_{\mathbf{\Omega_{k,p_{0}}}}(h)] \approx{} \mathbb{E}_{x,y}[l(h,y)\mathbf{1}_\mathbf{\Omega_{k,p_{0}}}(h)] \leq \mathbb{E}_{x,y}[l(h,y)] = R
\label{eq:4}
\end{equation}
The second term: $ L = \mathbb{E}_{x,y}[l(h * (1-f(z_k)) + a_k(h) * f(z_k),y) \mathbf{1}_{\mathbf{\Omega_{k,p_{0}}^{c}}}(h)]$ can be bounded as follows:
\begin{equation*}
\begin{split}
L &= \displaystyle{\mathop{\sum}_{k'}\mathbb{E}_{x,y \sim p(.|d=k')}[l(h * (1-f(z_k))} + a_k(h) * f(z_k),y) * \mathbf{1}_{\mathbf{\Omega_{k,p_{0}}^{c}}}(h)]p(d=k') \\
	& \leq \displaystyle{\mathop{\sum}_{k' \neq k}} A * \mathbb{E}_{x,y \sim p(.|d=k')} [\mathbf{1}_{\mathbf{\Omega_{k,p_{0}}^{c}}}(h)]P(d=k') \\
	& \quad + \mathbb{E}_{x,y \sim p(.|d=k)}[l(h * (1-f(z_k)) + a_k(h) * f(z_k),y) * \mathbf{1}_{\mathbf{\Omega_{k,p_{0}}^{c}}}(h)]P(d=k) \\
	&\leq A * \frac{\mathbb{\varepsilon}_{insep}}{p_{0}} + R_k * P(d=k)
\end{split}
\label{eq:5}
\end{equation*}

The last inequality is verified since:
\begin{equation}
\begin{split}
\displaystyle{\mathop{\sum}_{k' \neq k}} A * \mathbb{E}_{x,y \sim p(.|d=k')} [\mathbf{1}_{\mathbf{\Omega_{k,p_{0}}^{c}}} (h)]P(d=k') &\leq A * \displaystyle{\mathop{\sum}_{k' \neq k}} \mathbb{E}_{x,y \sim p(.|d=k') } [\frac{p_k(h)}{p_0}]P(d=k') \\
		& = \frac{A}{p_0} \mathbb{E}[\mathbf{1}_{d\neq k}p_k(h)] \\
		& \leq \frac{A}{p_0} \mathbb{\varepsilon}_{insep}
\end{split}
\end{equation}
because
$$ \mathbf{1}_{\mathbf{\Omega_{k,p_{0}}^{c}}}(h) \leq \frac{p_k(h)}{p_0} $$
Combining the inequality \ref{eq:4} and the inequality \ref{eq:5} we obtain the risk in general case of the adapted model $(M,a_k)$ $$R(M,a_k) \leq R + \frac{A * \mathbb{\varepsilon}_{insep}}{p_0} + P(d=k)R_k$$

\fyTodo{I sort of buy the equations but then so what ? We should may be compare with non-adaptive ?}

\end{document}


